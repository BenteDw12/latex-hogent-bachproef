%---------- Inleiding ---------------------------------------------------------

\section{Inleiding}%
\label{sec:inleiding}
\noindent
In de wereld van vandaag heeft elk bedrijf een website, zeker kleine ondernemingen zoals een privé-wellness gebruiken dit om klanten te informeren en de mogelijkheid te geven om afspraken te boeken en betalingen te verrichten. Voor bedrijven met een beperkte IT-kennis en middelen biedt een Content Management Systeem (CMS) een praktische oplossing voor het beheren van een website.
\\ \\
De keuze van een CMS is echter niet eenvoudig, elk systeem heeft zijn sterkte en zwaktepunten zoals kosten, gebruiksgemak, prestaties. Dit onderzoek richt zich op het identificeren van het meest geschikte CMS voor een specifieke privé-wellness op basis van deze factoren.
\\ \\
De eigenaar van een privé-wellness wil een nieuwe website laten bouwen die aantrekkelijk en gebruiksvriendelijk is voor klanten, maar ook eenvoudig te beheren. Het bedrijf heeft echter weinig kennis over CMS, waardoor het moeilijk is om de juiste keuze te maken. Dit onderzoek is gericht op IT-professionals en heeft als doel een systematische vergelijking van verschillende CMS’en uit te voeren om tot een technisch onderbouwde aanbeveling te komen. De centrale vraag van dit onderzoek luidt daarom: Welke CMS-oplossing biedt de meest gebruiksvriendelijke, kostenefficiënte en technisch geschikte basis voor het ontwikkelen van een website voor een privé-wellness?
\\ \\
Het resultaat van dit onderzoek omvat een uitgebreid rapport waarin de verschillende CMS's vergeleken en gedetaileerderd besproken worden. Dit rapport bevat een concrete aanbeveling voor de privé-wellness gebaseerd op de noden van dit product. Daarnaast worden er werkende prototypes van de website ontwikkeld met elk CMS, waarmee de aanbeveling wordt ondersteund.

%---------- Stand van zaken ---------------------------------------------------

\section{Stand van zaken}%
\label{sec:Stand van zaken}

% Voor literatuurverwijzingen zijn er twee belangrijke commando's:
% \autocite{KEY} => (Auteur, jaartal) Gebruik dit als de naam van de auteur
%   geen onderdeel is van de zin.
% \textcite{KEY} => Auteur (jaartal)  Gebruik dit als de auteursnaam wel een
%   functie heeft in de zin (bv. ``Uit onderzoek door Doll & Hill (1954) bleek
%   ...'')

Content management (CM) kan in simpele termen omschreven worden als het proces van het maken, verzamelen en structureren van informatie zodat ze kunnen worden opgeslagen, opgevraagd, gepubliceerd, bijgewerkt en hergebruikt kunnen worden op elke gewenste manier \autocite{Sunny2008}. Een content management systeem (CMS) is een software systeem dat gebruikt word voor content management, het bied een manier om grote hoeveelheden web-based informatie te managen zonder de moeilijkheden van programmeren.
\\ \\
Het onderzoek van \textcite{Khalil2024} onderzoekt WordPress, Drupal en Jooml op basis van verschillende requirements, hij concludeert dat elke CMS zijn eigen plus- en minpunten heeft. Bij het kiezen van een CMS moeten we dus per situatie apart bekijken welk systeem het beste voldoet aan de specifieke eisen van de organisatie. Terwijl WordPress uitblinkt in gebruiksvriendelijkheid en snelheid van implementatie, biedt Drupal meer flexibiliteit en schaalbaarheid voor complexe websites. Joomla wordt beschreven als een middenweg, geschikt voor gebruikers die behoefte hebben aan een balans tussen gebruiksgemak en technische mogelijkheden. \textcite{Khalil2024} benadrukt dat de keuze voor een CMS afhankelijk is van factoren zoals de grootte van de website, de technische expertise van de gebruikers, en de specifieke functionele vereisten van het project.
\\ \\
Wat ook nog een make or break kan zijn is de veiligheid. Volgens \textcite{MarianVladut2021} heeft WordPress ongeveer 18 miljoen installaties, dit is een marktaandeel van 64.2\%. Doordat er zoveel installaties hiervan zijn is WordPress een aantrekkelijk doelwit voor cyberaanvallen. Een kwetsbaarheid in de kernsoftware, plug-ins of thema's kan potentieel miljoenen websites tegelijk blootstellen aan risico's. Dit maakt het cruciaal dat WordPress-gebruikers regelmatig updates uitvoeren en adequate beveiligingsmaatregelen treffen om hun sites te beschermen.




%---------- Methodologie ------------------------------------------------------
\section{Methodologie}%
\label{sec:methodologie}

/

%---------- Verwachte resultaten ----------------------------------------------
\section{Verwacht resultaat, conclusie}%
\label{sec:verwachte_resultaten}

Mijn voorspelling voor dit onderzoek is dat Drupal de meest gebruiksvriendelijke en goedkoopste oplossing.