%---------- Inleiding ---------------------------------------------------------

\section{Inleiding}%
\label{sec:inleiding}
\noindent
In de wereld van vandaag heeft elk bedrijf een website, zeker kleine ondernemingen zoals een privé-wellness gebruiken dit om klanten te informeren en de mogelijkheid te geven om afspraken te boeken en betalingen te verrichten. Voor bedrijven met een beperkte IT-kennis en middelen biedt een Content Management Systeem (CMS) een praktische oplossing voor het beheren van een website.
\\ \\
De keuze van een CMS is echter niet eenvoudig, elk systeem heeft zijn sterkte en zwaktepunten zoals kosten, gebruiksgemak, prestaties. Dit onderzoek richt zich op het identificeren van het meest geschikte CMS voor een specifieke privé-wellness op basis van deze factoren.
\\ \\
De eigenaar van een privé-wellness wil een nieuwe website laten bouwen die aantrekkelijk en gebruiksvriendelijk is voor klanten, maar ook eenvoudig te beheren. Het bedrijf heeft echter weinig kennis over CMS, waardoor het moeilijk is om de juiste keuze te maken. Dit onderzoek is gericht op IT-professionals en heeft als doel een systematische vergelijking van verschillende CMS’en uit te voeren om tot een technisch onderbouwde aanbeveling te komen. De centrale vraag van dit onderzoek luidt daarom: Welke CMS-oplossing biedt de meest gebruiksvriendelijke, kostenefficiënte en technisch geschikte basis voor het ontwikkelen van een website voor een privé-wellness?
\\ \\
Het resultaat van dit onderzoek omvat een uitgebreid rapport waarin de verschillende CMS's vergeleken en gedetaileerderd besproken worden. Dit rapport bevat een concrete aanbeveling voor de privé-wellness gebaseerd op de noden van dit product. Daarnaast worden er werkende prototypes van de website ontwikkeld met elk CMS, waarmee de aanbeveling wordt ondersteund.

%---------- Stand van zaken ---------------------------------------------------

\section{Stand van zaken}%
\label{sec:Stand van zaken}

/

% Voor literatuurverwijzingen zijn er twee belangrijke commando's:
% \autocite{KEY} => (Auteur, jaartal) Gebruik dit als de naam van de auteur
%   geen onderdeel is van de zin.
% \textcite{KEY} => Auteur (jaartal)  Gebruik dit als de auteursnaam wel een
%   functie heeft in de zin (bv. ``Uit onderzoek door Doll & Hill (1954) bleek
%   ...'')



%---------- Methodologie ------------------------------------------------------
\section{Methodologie}%
\label{sec:methodologie}

/

%---------- Verwachte resultaten ----------------------------------------------
\section{Verwacht resultaat, conclusie}%
\label{sec:verwachte_resultaten}

Mijn voorspelling voor dit onderzoek is dat Drupal de meest gebruiksvriendelijke en goedkoopste oplossing.